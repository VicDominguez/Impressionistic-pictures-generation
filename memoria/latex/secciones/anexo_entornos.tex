\documentclass[../main.tex]{subfiles}

\renewcommand{\appendixname}{Anexos}

\begin{document}
\section{Hardware}
\subsection{Equipo personal del autor}
\begin{itemize}
    \item CPU: Intel Core i7 7700HQ @ 3,80 GHz, con cuatro núcleos, litografía de 14 nm y 6 MB de caché.
    \item RAM: 8GB DDR4-2400
    \item HDD: 2.5" SATA con 1TB de almacenamiento a @7200 rpm
    \item SSD: Crucial MX500 M.2 con 500GB de almacenamiento.
    \item GPU: nVidia GTX 1050 con 2GB de VRAM GDDR5. Dispone de la arquitectura Pascal y de 640 núcleos CUDA.
\end{itemize}
\subsection{Equipo contratado en Google Cloud Plataform}

\begin{itemize}
    \item Tipo de máquina: n1-standard-4 (4 vCPU, 15 GB RAM)
    \item GPU: 1x nVidia Tesla T4, con 16GB GDDR6 de VRAM. Dispone de 2560 núcleos CUDA, 320 núcleos \textit{Turing tensor} y de un ancho de banda de +320 GB/s
    \item SSD: 100GB
    \item Zona: us-west1-a 
\end{itemize}

\section{Software}

\begin{itemize}
    \item Sistemas Operativos: Ubuntu 18.04 LTS y Windows 10 versión 2004.
    \item Entorno de desarrollo de Python: PyCharm Professional Edition, versión 2020.2.3
    \item Python 3.6.9
    \item TensorFlow (y por extensión Tensorboard) 1.14 para el subsistema EnhanceNet y 2.2.1 para el sistema principal
    \item Keras 2.3.1
    \item Numpy 1.18.5
    \item Flask 1.1.2
    \item Driver nVidia 450.51.05
    \item Cuda 11.0
    \item cuDNN 7.6.5
    \item Pillow 8.0.0
\end{itemize}

\end{document}